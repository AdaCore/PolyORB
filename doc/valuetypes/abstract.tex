\begin{abstract}
\begin{center}
\begin{em}
Implementing value types in adabroker
\end{em}
\end{center}

\paragraph{}CORBA is an object-oriented, client-server model to allow
programs written in different languages and running on different
distant machines to interoperate. The specification consists of the
Interface Definition Language, used to define CORBA objects, and a
mapping of IDL to all the different programming languages. CORBA also
specifies how programs should interact with each other in a network
environment.

\paragraph{}Adabroker is an implementation of CORBA in Ada. It
consists of an IDL to Ada compiler, to translate the IDL definitions
into Ada types, and of a runtime, ``broca''. The runtime is in charge
of making all the low-level network interactions invisible to the
programmer. CORBA version 0pre1 supports CORBA 2.0.

\paragraph{}CORBA 2.3 was released by the OMG at the beginning of the
year 2000. Its main addition is the creation of a new object type,
called a ``value type''. Whereas CORBA interfaces were always passed
by reference, CORBA value types are always passed by value. This is a
new powerful feature of CORBA 2.3.

\paragraph{} The subject of the work presented here is to implement
support for valuetypes in adabroker. There are several issues. The
three main issues are the implementation of the new CORBA object
hierarchy, operation dispatching, and valuetype marshalling on the wire.

\paragraph{} This paper presents how the first two issues were
implemented, and how a prototype was made for the thirs one.

\end{abstract}


\begin{abstract}
\begin{center}
\begin{em}
Impl�mentation des types par valeur dans Adabroker
\end{em}
\end{center}

\paragraph{}CORBA est un mod�le client-serveur orient� objet, qui
permet � diff�rents programmes �crits dans diff�rents langages de
programmation et tournant sur diff�rentes machines d'interop�rer. La
sp�cification de CORBA contient deux choses. Tout d'abord la
d�finition de l'IDL (Interface Definition Language), qui permet de
d�finir des objets CORBA, ainsi que la traduction des d�finitions IDL
en objets des diff�rents langages. Ensuite, la norme pr�cise �galement
les protocoles utilis�s pour interop�rer sur le r�seau.

\paragraph{}Adabroker est une impl�mentation de la norme CORBA en
Ada. Il comprend un programme, idlac, et une librairie, broca. Idlac
est un compilateur qui permet de traduire les d�finitions d'objets IDL
en objets Ada. Broca est la librarie qui rend invisible �
l'utilisateur tous les appels r�seau. La version 0pre1 d'adabroker,
sortie en avril 2000, supporte la norme CORBA 2.0.

\paragraph{}La nouvelle version de la norme, CORBA 2.3, introduit
principalement un nouveau type d'objets : les types par valeur. Alors
que les interfaces sont toujours pass�es en param�tre par r�f�rence,
les types par valeurs sont pass�s, comme leur nom l'indique, par
valeur. A part cela, ils poss�dent quasiment toutes les
caract�ristiques des interfaces.

\paragraph{}Le sujet de ce m�moire est l'impl�mentation des types par
valeur dans Adabroker. Les trois principales t�ches sont :
l'impl�mentation de la nouvelle hi�rarchie d'objets d�finie dans CORBA
2.3, l'appel de fonctions sur les types par valeur, et l'encodage des
types par valeur sur le r�seau.

\paragraph{}Ce m�moire pr�sente la fa�on dont les deux premi�res
t�ches ont �t� impl�ment�es, et propose un prototype d'impl�mentation
de l'encodage des types par valeur.

\end{abstract}