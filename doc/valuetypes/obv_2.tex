\chapter{New features of CORBA 2.3}

%%%%%%%%%%%%%%%%%%%%%%%%%%%%
%%% A new object hierarchy
%%%%%%%%%%%%%%%%%%%%%%%%%%%%
\section{A new object hierarchy}


%%% A new Architecture
%%%%%%%%%%%%%%%%%%%%%%%%%%%%
\subsection{A new architecture}
The ada object hierarhy in CORBA 2.3 is represented in
figure~\ref{fig::simple_hierarchy}.


\vspace{2mm}
\begin{figure}[!htbp]
    \includegraphics[width=1.1\textwidth,clip=]{simple_hierarchy.ps}
  \caption{The standardized object hierarchy.}
  \label{fig::simple_hierarchy}
\end{figure}

\begin{description}
\item[CORBA.AbstractBase] is the root of all references to interfaces or
valuetypes. It is responsible for reference counting and automatic
memory deallocation. It basically refers to an instance of CORBA.Impl.Object.

\item[CORBA.Impl.Object] is the root of all interface and valuetype implementations.
\end{description}

\subsection{Local interfaces}
\paragraph{}Local interfaces are special interfaces. There is no way
to get a remote reference to a local interface. Thus, when you call a
subprogram on such an interface, you are guaranteed that no remote
call will be performed. They are currently not supported in AdaBroker.

\subsection{Valuetypes}
\paragraph{}Here is the OMG definition of a valuetype : ``The basic notion is
relatively simple. A valuetype is half way between a regular interface
and a struct''.

\paragraph{}Like structs, valuetypes contains members, called state
members. These state members can be public or private.
Thet are always local. There is no way of
performing a remote call on a valuetype. Thus, to handle a valuetype,
a program has to contain the correspondind object code. Since, they
are always local, valuetypes are always passed by value to
subprograms. Each time a client makes a remote invocation on a server,
for a subprogram that takes a valuetype as parameter, a copy of the
valuetype is made, and both the server and the client have to contain
the code for the valuetype used in order for the call to succeed.

\paragraph{} However, valuetypes are more expressive than structs
since they can represent any arbitrary complex structure, like graphs
or cycles. They also support null value semantics.

\paragraph{}On the other hand, valuetypes are object oriented. There
are abstract valuetypes. They cannot have any state member, and are
therefore a bunch of subprograms for which you are guaranteed that no
remote call will be performed. Concrete valuetypes
(as opposed to abstract valuetypes) may inherit from at most one
concrete valuetype. Abstract valuetypes however, support multiple inheritance.

\paragraph{}An advanced feature of valuetypes is that they can
``support'' interfaces. A valuetype that supports an interface
corresponds to an implemetation of this interface. Consider the following
example:

\begin{verbatim}
interface Item {
	real getPrice();
};

valuetype ItemValue supports Item {
	private full_price;
	private disount_rate;
};
\end{verbatim}
\paragraph{}The Item interface only specifies that all items have a price. The
ItemValue valuetypes is more precise, and specifies a way of
implementation, with a full price and a discount rate.

\paragraph{}CORBA 2.3 also introduces the ``valuebox'' type. It is a
degenerated case of valuetype with only one state member and no
method. Its semantic is similar to that of typedef, but also includes
null value semantics.


\subsection{Abstract interfaces}
\paragraph{}The semantics of abstract interfaces have changed a bit since CORBA
2.0. As a matter of fact, abstract interfaces can refer arbitrarily to
``regular'' interfaces or to valuetypes. This is very useful if one
wants to determin at runtime if a subprogram parameter will be passed
by value or by reference. Consider what happens if the formal
parameter of function foo is an abstract interface. If the actual
parameter is a regular interface, then it will be passed by
reference. However, if it is a valuetype, then it will be passed by
value.

\paragraph{}
Abstract interfaces are not present on the diagram.
They inherit
directly from \\
CORBA.AbstractBase.Ref.


%%% Revised mapping for interfaces
%%%%%%%%%%%%%%%%%%%%%%%%%%%%%%%%%%%
\section{A revised ada mapping for interfaces}

\subsection{Helper packages}
\paragraph{}For each IDL interface, the OMG specifies that 3 ada packages have to
be generated. One package for the reference type, one for the
implementation type, and a third one, new in CORBA 2.3, the helper
package.

\paragraph{}The Helper package contains widening and narrowing
functions, called ``To\_Ref'', to cast object references in more
general or more specific reference types.

\subsection{abstract interfaces}
\paragraph{}The mapping for abstract interfaces has changed since they
now can refer to valuetypes, that did not exist in CORBA 2.0. No
package is generated for their implementation, since they cannot be
instanciated. Moreover, they inherit directly from CORBA.AbstractBase.Ref.

\subsection{delegating servants}
\paragraph{}CORBA 2.0 specifies that an ada Foo.Impl package has to be generated
for every IDL interface Foo. This Foo.Impl package is the place where
the CORBA programmer has to implement the subprogram bodies. This may
be a heavy constraint when somedy has some code that he wants to make
available within CORBA. With this model, he would have to rewrite, or
at least copy, all his code in the automatically generated Foo.Impl
package.

\paragraph{}CORBA 2.3 introduces the Foo.Delegate package. This is
just a proxy object whose role is to receive calls made from clients,
to redirect them to the actual implementation, to get the response,
and to send it back to the client.


%%%%%%%%%%%%%%%%%%%%%%%%%%%%%%%%%%%%
%%% Advanced features of valuetypes
%%%%%%%%%%%%%%%%%%%%%%%%%%%%%%%%%%%%
\section{Advanced features of valuetypes}

\subsection{Initializers}
\paragraph{} Since they have specified fields, valuetypes must have
specific functions to be initialised. These functions are called
initialisers and are specified in IDL with the keyword
``factory''. These initialisers are mapped to ada functions that
return a reference to a newly created object.

\subsection{Forward valuetypes}
\paragraph{}In IDL, valuetypes can be forward declared, and therefore
2 different valuetypes can point to each other. Since two ada packages
cannot ``with'' each other, an intermediate type has to be used to
allow such structures. These are called forward valuetypes. They are
similar to forward interfaces.

\subsection{Truncatable valuetypes}
\paragraph{} We said earlier that valuetype implementations are always
local, and that valuetypes are always passed by value. Therefore, when
a client or a server receives an instance of a valuetype from the
wire, it needs the corresponding object code for this
valuetype. What happens if the server does not have the corresponding
object code ?

\paragraph{}
Consider the Folowing example:

\begin{verbatim}
valuetype Drink {
	public long quantity;
};
valuetype FruitJuice : Drink{
	public string fruit;
};
interface Foo {
	void killThirst(in Drink d);
};
\end{verbatim}
\paragraph{}
Consider a server that has the code for Drink and Foo. Consider a
client, that has the code for Foo, Drink and SoftDrink. If the client
gets a reference to an instance on the server of interface foo, and
calls its killThirst method with a parameter SoftDrink.

\paragraph{}The client will send on the wire a parameter of type
FruitJuice, with the quantity and fruit. The server will not be able
to create a new instance of FruitJuice, since it does not know what a
FruitJuice is. Therefore, the call will fail.

\paragraph{}However, there is a way to make the call succeed, if
FruitJuice had been declared like this.

\begin{verbatim}
valuetype FruitJuice : truncatable Drink {
	public string fruit;
};
\end{verbatim}

The truncatable keyword means that if a program does not have the
implementation code od a FruitJuice, it is allowed to cast it into a
Drink, and to perform the call. The keyword ``truncatable'' must be
used carefully, since it is not always safe from a semantics point of
view to drop some fields of a valuetype.

\subsection{Custom valuetypes}
\paragraph{}
As we said earlier, valuetypes are passed by value and a new instance
is created each time a subprogram is called. Therefore, CORBA 2.3
defines a standard way to send valuetypes on the wire (using the GIOP
protocol). It simply sends the valuetype's fields one by one. To
handle recursive constructs like graphs, CORBA specifies that
valuetypes are sent only once on the wire. If the same value is
encountered twice when traversing the graph, only a pointer to the
position in the stream where it had been marshalled earlier is sent.

\paragraph{} This protocol is guaranteed to work all the times, but it
may be very heavy for valuetypes that have specific
characteristics. In such a case, the CORBA programmer can define the
valuetype as Custom. It means that he will provide the necessary
functions to send and receive a valuetype to/from the wire. This has
to be used very carefully however, because in this case the programmer
is responsible for consistency when he writes the code.
