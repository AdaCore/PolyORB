%%% AdaBroker boilerplate presentation slides
%%% $Id$

\documentclass[a4,slidesec]{seminar}

\newif\ifpdf
\ifx\pdfoutput\undefined\pdffalse\else\pdftrue\fi

\usepackage{fancyhdr}
\ifpdf\else
\usepackage{epsfig}
\fi
\usepackage{amsmath}
\usepackage[latin1]{inputenc}

\newcommand{\nom}[1]{{#1}}
%% \newcommand{\strong}[1]{\textbf{#1}}
\newcommand{\dcl}{\textsc{dcl}}
\newcommand{\unix}{\textsc{Unix}}
\newcommand{\ab}{AdaBroker}
\newcommand{\ada}{Ada~95}
\newcommand{\strong}[1]{\textbf{#1}}
\newcommand{\id}[1]{\texttt{#1}}
\newcommand{\bibname}{}

% Ada syntax
\newcommand{\asynt}[1]{\mbox{\textsf{#1}}}
% IDL syntax
\newcommand{\isynt}[1]{\mbox{$<$\textsf{#1}$>$}}

% Fonte Computer Roman Bold Caps-and-Small-Caps

\DeclareFontShape{OT1}{cmr}{b}{sc}
    {
      <5><6><7><8><9><10><10.95><12>
      <14.4><17.28><20.74><24.88> cmbcsc10
      }{}
\DeclareFontShape{OT1}{cmr}{bx}{sc}
      {<->ssub * cmr/b/sc}{}


\title{\large\sffamily
\ab{}\\
An open source CORBA suite\\
for \ada{}}

\author{Thomas \nom{Quinot}}
\date{}

\fancyhf{}
\fancyfoot[L]{\small\sffamily\theslideheading}
\fancyfoot[R]{\small\sffamily\thepage}
\renewcommand{\footrulewidth}{\headrulewidth}
\renewcommand{\headrule}{\hbox to \headwidth {%
\kern 0pt%
\leaders\hrule height 0.5ex depth -0.4ex\hfill\kern 0pt%
\hspace{1ex}%
\raisebox{-0.4cm}{
\ifpdf\pdfimage height 0.8cm {telecom-nb.pdf}
\else\epsfig{file=telecom-nb.eps,height=0.8cm}
\fi
}%
\hspace{1ex}%
\rule[0.4ex]{2em}{0.1ex}}}

\pagestyle{fancy}

\slidewidth 250mm
\textwidth 270mm

\begin{document}

\input{seminar.bug}
\headwidth=\textwidth
\addtolength{\headheight}{0.8cm}
\addtolength{\footheight}{24pt}
\addtolength{\slideheight}{-\headheight}
%\addtolength{\parskip}{0.5ex}
\renewcommand{\slideleftmargin}{1cm}
\renewcommand{\sliderightmargin}{1cm}

\slideframe{none}

\begin{slide}
\maketitle
\end{slide}

%%%% Introduction

\begin{slide}
\slideheading{CORBA}

Industry standard specification of platform for distributed object-oriented
applications.

Reference model:
\begin{itemize}
\item IDL and language mappings;
\item stubs, skeletons, object adapters;
\item ORB;
\item protocols;
\item services:
  \begin{itemize}
  \item Common Object Services, Common Facilities;
  \item Domain Interfaces.
  \end{itemize}
\end{itemize}

\end{slide}

\begin{slide}
\slideheading{CORBA overview}

\begin{center}
\ifpdf\pdfimage width 8cm {corbaglobal.pdf}
\else\psfig{file=corbaglobal.eps,width=8cm}
\fi
\end{center}

\end{slide}


\begin{slide}

\slidesubheading{CORBA development process}

\begin{enumerate}
\item Objects are described by an IDL contract.
\item IDL compiler generates stubs and skeletons in
  target languages.
\item Object implementations are created by implementation
  of skeleton's methods.
\item Clients can use stubs to invoke operations on
  (possibly remote) objects.
\end{enumerate}

\end{slide}

\begin{slide}
\slidesubheading{Why CORBA?}

\begin{itemize}
\item Distribution is a requirement
\item Distributed OO middleware hides some of its complexity/tediousness
\end{itemize}

\slidesubheading{Applications examples}
\begin{itemize}
\item scientific data acquisition (IN2P3);
\item telecoms (Iridium);
\item desktop integration (GNOME).
\end{itemize}
\end{slide}

\begin{slide}
\slideheading{The Ada~95 standard mapping}


\centering
\begin{tabular}{l|l}
\textbf{IDL} & \textbf{Ada} \\ \hline
modules & packages \\
types & types \\
interfaces & tagged types \\
interface inheritance & tagged type derivation \\ \hline
multiple inheritance & copy of methods \\ \hline
sequences & generics
\end{tabular}

\end{slide}

\begin{slide}
\slideheading{\ab}

\begin{itemize}
\item CORBA tool suite
\item Free software (GPL)
\item Written in \ada{}
\item Targeted at \ada{}
\end{itemize}
\end{slide}

\begin{slide}
\slidesubheading{\ab{} features}

\hfill
\parbox[t]{0.45\textwidth}{%
\begin{center}
\textbf{Core ORB features}
\end{center}

\begin{itemize}
\item CORBA 2.0 ORB
\item GIOP 1.0
\item IIOP 1.0
\item Portable Object Adapter
\item IDL $\rightarrow$ Ada compiler
\end{itemize}
}%
\hfill%
\parbox[t]{0.45\textwidth}{%
\begin{center}
\textbf{Services}
\end{center}
\begin{itemize}
\item Interface Repository
\item Dynamic Invocation
\item COS Naming
\item COS Events
\item COS Time
\item GNOME Applet Library
\end{itemize}
}

\end{slide}

\begin{slide}
\slideheading{Project history}

Development as a succession of student projects at ENST.

\begin{tabular}{l|ll}
\textbf{When} & \textbf{What} & \\ \hline
1999Q1 & 0.8 & based on omniORB and Sun parser \\
1999Q2 &  0.9 & new ORB, POA, still Sun parser \\
1999Q3-present & 1.0 & New compiler, many new features \& services
\end{tabular}

\end{slide}

\begin{slide}

\slidesubheading{The first C++-based version}

Ada binding to omniORB's communication library.

CORBA Object management in Ada, compliant with standard
binding.

Problems encountered:
\begin{itemize}
\item Ada $\longleftrightarrow$ C++ interfacing.
  Thread libraries incompatibilities.
\item C++ IDL front-end from Sun: hardly maintainable.
  Licensing issues.
\end{itemize}

\end{slide}

\begin{slide}

\slidesubheading{Second version: all-Ada ORB with a POA}

CIAO (ENST research project: CORBA/DSA gatewat
generator) required a POA.

Reimplementation of a complete ORB in Ada.

Still used Sun's front-end.
\end{slide}

\begin{slide}

\slidesubheading{3rd version: an Ada front-end for new services}

Development of new services and functionalities: objects-by-value (CORBA 2.3),
interface repository, dynamic invocation.

Required IDL compiler support; Sun front-end not suitable for extension.

Reimplementation of the IDL-to-Ada compiler in Ada.

\end{slide}

\begin{slide}

\slideheading{CIAO}

An application developed with AdaBroker: CIAO.

CORBA Interface for Ada~95 distributed Objects.

Gateway generator for Ada DSA services.

Based on AdaBroker and ASIS.

\end{slide}

\begin{slide}
\slidesubheading{Current status}

\begin{itemize}
\item Students have completed ENST cursus.
\item Maintained as a research testbed at ENST.
\item Community has demonstrated interest in the product.
\item Seeking further involvement of users with development.
\end{itemize}

\end{slide}

\end{document}
